\documentclass[fontsize=11pt]{article}
\usepackage{amsmath}
\usepackage[utf8]{inputenc}
\usepackage[margin=0.75in]{geometry}
\usepackage{graphicx}

\usepackage{enumerate}
\usepackage{url}

\title{CSC111 Project Proposal: How can we build an AI for playing and winning a Five in a Row game?}
\author{Qianning Lian, Xueqing Zhai, Zeyang Ni, Aiden Lin}
\date{Tuesday, March 16, 2021}

\begin{document}
\maketitle

\section*{Problem Description and Research Question}

\textbf{Our research question is: How can we build an AI for playing and winning a Five in a Row game?}

The game Five in Row is a popular traditional board game in east Asia. Many Asians have played and enjoyed the game with their friends and family in their childhood. However, because of COVID-19, people are not allowed to gather and play games together with their friends. Therefore, we decided that we want to build a five in a row Pygame application that allows humans to play with an AI, letting them to enjoy the game even in quarantine. We wish to bring people happiness during difficult times like these. 

We are building an AI for playing a 2 player board game called Five in Row. Two players, black and white, play on a board of size 15 by 15. Players place their piece on the board in turn, black taking the first move, and must place the piece on the center of the board. The goal for each player is to link their pieces in a row of 5 and/or to stop the other player from doing so. The player who first links their pieces in a row of 5 wins. The row of five could be horizontal, vertical, or diagonal. The game ends in a draw when there is no space for another piece to be placed. 

Learn more about the game with the following link: 
\url{https://en.wikipedia.org/wiki/Gomoku}

Our idea of building an AI is inspired by the gametree class and minichess game class in assignment 2. This game is different and slightly more complicated than the minichess game in A2, but all possible moves could still be represented by a tree structure. Because of the different rules and winning strategy for this game, our gametree will have different attributes from the gametree in A2, but generally would have a similar structure.


\section*{Computational Plan}
Our project will use tree to represent possible moves on the chess board. The tree will be generated after each move of the opponent, with a certain depth (around 5). 
By giving the score to every possible step that players can place, the AI selects the largest score of one branch which is most beneficial to winning the game. The scores are calculated based on the situation on the board - the positions of pieces. It has similar idea and computation as the A2 white\_win\_probability attribute. After generating all possible moves for one subtree, the scores are calculated for each leaf and would be passed up to their parent, until the root is reached. Upon that, we also want to use the alpha-beta pruning algorithm to reduce run time. \\

\textbullet \hspace{2mm} \texttt{Alpha-beta algorithm:}\\
\begin{figure}[h]
    \centering
    \includegraphics[width = 10cm, height =4cm]{tree1.jpg}
    \caption{Calculating score for 1st subtree}
    \label{fig:my_label}
\end{figure}

Looking at the graph above. Suppose the score for leaf A is -1, for leaf B is 4. We assume that the opponent would take the step that results in worst case of our AI's game, so the node above A and B would have value -1 (minimum of all scores of its subtrees). \\
Until now, the algorithm is the same as in A2. \\

\begin{figure}[h]
    \centering
    \includegraphics[width = 10cm, height =4.5cm]{tree2.jpg}
    \caption{Calculating score for 2st subtree}
    \label{fig:my_label}
\end{figure}

Suppose when generating C, C has a score of -3. Then, since we know that out opponent chooses the worst case, the node above it must have score $\leq -3$. We notice that this score is already worse than the first subtree (with score -1). So immediately, we can delete this whole branch and move to the next one. 

\begin{figure}[h]
    \centering
    \includegraphics[width = 10cm, height =5cm]{tree3.jpg}
    \caption{Calculating score}
    \label{fig:my_label}
\end{figure}

This algorithm would reduce our run time a lot, which is important for our project, since the game board and the number possible moves for a game of Five in Row is much larger than the minichess game we looked at in A2. \\

\textbullet \hspace{2mm} \texttt{MiniMax algorithm:}\\

\begin{figure}[h]
    \centering
    \includegraphics[width = 8cm, height =3.5cm]{minimax.jpg}
    \caption{minimax algorithm}
    \label{fig:my_label}
\end{figure}

For This game, we will always take black player(the firsi player to move) as Maximizing player(always choose the situation with highest score when possible) and white player as Minimizing player(always choose the situation with lowest score when possible).\\

When the situation score are evaluated at each leaf, the current player would choose the score that favors itself the most and pass that value to its ancestry, and among all the ancestries, their ancestry would then pick another value that favor itself most repeatedly. \\

\\
The project consists of several steps:

\begin{enumerate}[I]
    \item. Get every possible moves that is worth taking.
    \item. For each potential move, estimate next player's potential next move and evaluate opponent's move. 
    \item. Repeat step I and II according to the given depth and form a tree.
    \item. evaluate the score(attack score and defence score) of each leaf and pass it to its ancestry according to Minimax algorithm.
    \item. Using alpha-beta pruning searching to accelerate the process.
    \item. pick the move that has the most favorable score to current player.
    \item. Display the game in a tkinter window.
    
\end{enumerate}

\textbullet \hspace{2mm} \emph{Find all possible moves:}\par

Find all worth taking moves. By worth taking we mean that the next move should contribute to your success in some way. And there are several situations that can be evaluated as worth taking.\\

\begin{enumerate}[i]
    \item One in a row
    \item Two in a row
    \item Three in a row 
    \item Four in a row 
    \item Five in a row
\end{enumerate}
By in a row we mean that it can potentially become five in a row vertically, horizontally and diagonally.\\

\textbullet \hspace{2mm} \emph{Evaluation Part:}\par
The evaluation part is designed to evaluate the score of current situation based on following criteria.\\

\begin{align*}
10 &-- \text{Blank(useless)}\\
35 - (n * 5) &-- \text{one in a row with n obstacles, n $\leq$ 2. n refers to the directions that are blocked.}\\
800 - (n * 200) &-- \text{two in a row with n obstacles.}\\
2000 - (n * 400) &-- \text{three in a row with n obstacles.}\\
10000 - (n * 4000)& -- \text{four in a row with n obstacles.}\\
\infty & -- \text{five in a row with n obstacles.}
\end{align*}

The evaluation part is designed on the rules that black player moves first:\\

\textbullet \hspace{2mm} The line up of black pieces evaluate to positive score representing the favorable situation to black player and detrimental to white player.\\

\textbullet \hspace{2mm} the line up of white pieces evaluates to negative score representing the favorable situation to white player and detrimental to black player. \\

\textbullet \hspace{2mm} The final score is gained by adding up the two scores, black player would be inclined to choose the move with higher score and white player would be inclined choose the move that has the lowest score.\\

\textbullet \hspace{2mm} By Minimax algorithm and Alpha-beta searching, current player then pick the move that is most favorable to itself.\\

\textbullet \hspace{2mm} Displaying the game: \texttt{Tkinter} Library: \\

In this project, we will use Tkinter, a library that allows us to build a simple GUI. We would display visualizations of the board and the pieces and react to user mouse activities (click, drag and drop, etc). \\

Some functions we might use: 
\begin{enumerate}
    \item. \texttt{tkinter.colorchooser}: use this function to let user choose the colour they want to be, black or white. 
    
    \item. \texttt{tkinter.messagebox}: can be used to confirm if the user wants to place the piece in the spot they clicked. 
    
    \item. \texttt{bind}: look for valid mouse click actions that triggers callback functions. 
    
    \item. \texttt{print\_event}: prints out the event and the mouse action. Can be used to teach user how to play the game. 
    \item. The window manager \texttt{wm}: can be used to construct title, icon, buttons, etc. 
    
    \item. \texttt{tkinter.image}: visualization for buttons, board, pieces etc. 
    
\end{enumerate}

\section*{References}

\hspace*{5mm}“Alpha–Beta Pruning.” Wikipedia, Wikimedia Foundation, 6 Mar. 2021,
\url{https://en.wikipedia.org/wiki/Alpha–beta\_pruning}.\\

Dufour, Bruno, and Wen Hsin Chang. “An Introduction to Tkinter.” An Introduction to Tkinter, \url{www.cs.mcgill.ca/~hv/classes/MS/TkinterPres/}.\\

“Gomoku.” Wikipedia, Wikimedia Foundation, 13 Mar. 2021, 
\url{https://en.wikipedia.org/wiki/Gomoku}.\\

Lague, Sebastian. Algorithms Explained – Minimax and Alpha-Beta Pruning. YouTube, YouTube, 20 Apr. 2018, \url{www.youtube.com/watch?v=l-hh51ncgDI}.\\

“Graphical User Interfaces with Tk.” Graphical User Interfaces with Tk - Python 3.9.2 documentation, 13 Mar. 2021, \url{https://docs.python.org/3/library/tk.html}.

% NOTE: LaTeX does have a built-in way of generating references automatically,
% but it's a bit tricky to use so we STRONGLY recommend writing your references
% manually, using a standard academic format like APA or MLA.
% (E.g., https://owl.purdue.edu/owl/research_and_citation/apa_style/apa_formatting_and_style_guide/general_format.html)

\end{document}
